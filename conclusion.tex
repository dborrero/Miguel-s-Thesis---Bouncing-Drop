\chapter*{Conclusion}
         \addcontentsline{toc}{chapter}{Conclusion}
	\chaptermark{Conclusion}
	\markboth{Conclusion}{Conclusion}
	\setcounter{chapter}{4}
	\setcounter{section}{0}
	
%Here's a conclusion, demonstrating the use of all that manual incrementing and table of contents adding that has to happen if you use the starred form of the chapter command. The deal is, the chapter command in \LaTeX\ does a lot of things: it increments the chapter counter, it resets the section counter to zero, it puts the name of the chapter into the table of contents and the running headers, and probably some other stuff. 

%So, if you remove all that stuff because you don't like it to say ``Chapter 4: Conclusion'', then you have to manually add all the things \LaTeX\ would normally do for you. Maybe someday we'll write a new chapter macro that doesn't add ``Chapter X'' to the beginning of every chapter title.

The question we sought to answer was: How does tunneling probability of a walking droplet change with the depth of oil above the barrier? Our results showed that tunneling is highly sensitive in this system, and even changes in height on the order of fractions of a millimeter are enough to radically influence the proportion of tunneling droplets. Using a barrier of width $e=3.0~\mathrm{mm}$, it was found that a depth of $h=1.5~\mathrm{mm}$ produced tunneling in every interaction, while at $h=1.0~\mathrm{mm}$ there was no tunneling. In the middle of this range was a sweet spot of $h=1.25~\mathrm{mm}$ where tunneling appeared probabilistic, but still somewhat dependent upon droplet diameter and droplet velocity. Our data suggests that for a given barrier, a droplet with a higher momentum is more likely to tunnel than a droplet with lower momentum. For a droplet of a constant diameter, tunneling occurs at higher values of $h$. The proportion of tunneling events decreases as the value of $h$ decreases.

The main limitations in this investigation had to do with consistency of parameters between trials, and dearth of data points. The damping of the shaker made keeping consistent conditions in every trial difficult, and limited the number trials and interactions that were filmed. A better shaker would have significantly improved these things. For example, the shaker modeled in \rf{shaker}, shakes the whole tray at the same time and with same amplitude for hours. These shakers of course, cost more than the budget allowed for. Another difficulty was in measuring the height of the oil within a trial. When removing each barrier, a certain amount of oil was lost. While this value was estimated, it still was a source of uncertainty and it introduced contamination from the pliers into the oil (making coalescence likely).

\subsection*{Future Work}
An extension of this topic would benefit from using barriers of height $2.90~\mathrm{mm}$ and $3.10~\mathrm{mm}$ in addition to those used in this investigation, as this would have allowed for greater resolution in the tunneling probabilities as a function of $h$. Limitations in the shaker meant that testing more than three barriers using a single droplet became exceedingly difficult as the shaker's performance decayed, but an improved set up would make it possible. 

Another avenue of study would be looking at how memory affected tunneling. For a constant barrier and a constant droplet, does adjusting memory affect the probability of tunneling? Because quantum-like behaviors emerge at higher memories we might expect these tunneling properties to exist only at these higher memories. This experiment would be relatively easy to carry out since it uses a single barrier.

Finally, a few words of advice for those seeking to recreate the experiment: take your time in setting up the device, ensuring that the tray is level and that it vibrates vertically. Invest in good silicone oil, and do your best to limit any contamination of the oil. Finally, there is an accelerometer out there that does what you need, your task is simply to find it. 