\chapter{Introduction to Tunneling in QM}	


\section{Wavefunction}	
In classical mechanics, one can describe a particle with six variables: three position indicators ($x$, $y$, $z$), and three momentum indicators ($p_x$, $p_y$, $p_z$). One can find a function to represent variable by making multiple measurements across time, and discover an expression $x(t)$ that describes the particles location in space along the x dimension at time $t$. When all six variables can be described by a function, one can describe the position of the particle at any time.  

In quantum mechanics however, Heisenberg's uncertainty priciple limits the knowledge of position and momentum: 
$$\sigma_{x} \sigma_{p} \geq \hbar/2. $$
This equation states that as one knows more about the position of the system, then one loses knowledge about the momentum of the system, and vis versa. It's a tradeoff inherent to the nature of the system, not due to experimental deficiences. And it means that the strategy of finding equations for position and momentum of a system are no longer possible. If a particle has no exact position, can one represent where it might be? One can use a wave describing the probability of finding the particle at that position. The square root of this wave is called the wavefunction, and is represented by $\Psi(x, t)$.


\section{Schroedinger's Equation}

The wavefunction $\Psi(x, t)$ is goverened by the Schroedinger equation:

$$i\hbar\frac{\partial}{\partial t} \Psi(x,t) = \left [ \frac{-\hbar^2}{2m}\frac{\partial^2}{\partial x^2} + V(x,t)\right ] \Psi(x,t) $$
where $V(x,t)$ is the given potential, $m$ is the mass of the particle, and $\hbar$ is the reduced Planck's constant. For potentials that do not change in time, one can use separation of variables to arrive at the time-independant Schroedinger equation
$$E \Psi(x,t) = \left [ \frac{-\hbar^2}{2m}\frac{\partial^2}{\partial x^2} + V(x)\right ] \Psi(x,t)$$
where $E$ is the total energy. The goal is to find $\Psi(x,t)$ knowing the potential $V(x)$.




\section{Barrier Potential}

\section{Energies}
    \subsection{$E = V_0$}
    \subsection{$E > V_0$}
    \subsection{$E < V_0$}
    
    
    
    
\section{Physics}

Many of the symbols you will need can be found on the math page (\url{http://web.reed.edu/cis/help/latex/math.html}) and the Comprehensive \LaTeX\ Symbol Guide (enclosed in this template download).  You may wish to create custom commands for commonly used symbols, phrases or equations, as described in Chapter \ref{commands}.
