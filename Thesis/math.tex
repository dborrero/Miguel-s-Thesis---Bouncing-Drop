\chapter{Introduction to Tunneling in QM}	


\section{Wavefunction}	
In classical mechanics, one can describe a particle with six variables: three position indicators ($x$, $y$, $z$), and three momentum indicators ($p_x$, $p_y$, $p_z$). One can find a function to represent variable by making multiple measurements across time, and discover an expression $x(t)$ that describes the particles location in space along the x dimension at time $t$. When all six variables can be described by a function, one can describe the position of the particle at any time.  

In the standard, Copenhagen interpretation of quantum mechanics however, Heisenberg's uncertainty principle limits the knowledge of position and momentum: 
$$\sigma_{x} \sigma_{p} \geq \hbar/2. $$
This equation states that as one knows more about the position of the system, then one loses knowledge about the momentum of the system, and vis versa. It's a tradeoff inherent to the nature of the system, not due to experimental deficiencies. And it means that the strategy of finding equations for position and momentum of a system are no longer possible. If a particle has no exact position, can one represent where it might be? One can use a wave describing the probability of finding the particle at that position. The square root of this wave is called the wavefunction, and is represented by $\Psi(x, t)$.


\section{Schroedinger's Equation}

The wavefunction $\Psi(x, t)$ is governed by the Schroedinger equation:

$$i\hbar\frac{\partial}{\partial t} \Psi(x,t) = \left [ \frac{-\hbar^2}{2m}\frac{\partial^2}{\partial x^2} + V(x,t)\right ] \Psi(x,t) $$
where $V(x,t)$ is the given potential, $m$ is the mass of the particle, and $\hbar$ is the reduced Planck's constant. For potentials that do not change in time, one can use separation of variables to arrive at the time-independent Schroedinger equation
$$E \Psi(x,t) = \left [ \frac{-\hbar^2}{2m}\frac{\partial^2}{\partial x^2} + V(x)\right ] \Psi(x,t)$$
where $E$ is the total energy. The goal is to find $\Psi(x,t)$ knowing the potential $V(x)$.




\section{Barrier Potential}

Given a potential barrier of width $2a$ and height $V_0$, we can find a probability of a particle reflecting back from the barrier and also the probability of the particle tunneling through the barrier. Since there are only two options, we know that the probability of reflection $R$ and the probability of transmission $T$ must sum to 1.

$$T + R = 1$$

This is a time independent situation, so we can get away with using the time independent schroedinger equation. We can find $R$ by solving this equation in three separate regions ($x < 0$, $0 < x < 2a$, $2a < x$), and piecing them together using continuity of $\psi$ and the first derivative of $\psi$ at the two boundaries. Using the value of $R$, we can easily compute the value of $T$. 

\subsection{Continuity of $\psi$}

Starting with the continuity of $\psi$ at 0, we write that $\psi_I(0)~=~\psi_{II}(0)$ which gives us:  
\begin{equation}
A + B = D,
\label{front}
\end{equation}
while at the other end of the barrier we write $\psi_{II}(2a)~=~\psi_{III}(2a)$ giving:
\begin{equation}
C \mathrm{sin}(2 k_2 a) + D \mathrm{cos}(2 k_2 a) = F e^{2a i k_1}.
\label{end}
\end{equation}



\subsection{Continuity of $\psi'$}

At the front of the barrier, our expression $\psi_I'(0) = \psi_{II}'(0)$ simplifies to:
\begin{equation}
C = \frac{i k_1}{k_2}(A-B),
\label{frontp}
\end{equation}
and finally, our last coninuity equation is written as $\psi_{II}'(2a) = \psi_{III}'(2a)$ and gives us:  
\begin{equation}
C \mathrm{cos}(2 k_2 a) - D \mathrm{sin}(2 k_2 a) =\frac{i k_1}{k_2} F e^{2a i k_1}.
\label{endp}
\end{equation}
 
\subsection{Finding $R$}

We know that the probability of reflection is the wave being reflected (represented by our coefficient $B$) divided by the total incoming wave (the coefficient $A$).

$$R = \left|\frac{B}{A}\right|^2$$

Finding $R$ requires that we solve for $A$ in terms of $B$. We can do this by plugging \refeq{front} into \refeq{frontp} to eliminate $C$ and $D$ and make a new equation. Plug this equation into \refeq{end}, and then again into \refeq{endp} to find two new equations. Using these two new equations, we find an expression: 

$$ A~\frac{k_1^2-k_2^2}{k_2^2}~\mathrm{sin}(2k_2a) = B~\left[\frac{k_1^2+k_2^2}{k_2^2}~\mathrm{sin}(2k_2a) + \frac{2 i k_1}{k_2}~\mathrm{cos}(2k_2a)\right]$$

We can rearrange this expression to solve for $\frac{B}{A}$, and squaring that to find $R$. Using the relation $T = 1 - R$ we find that:

\begin{equation}
T^{-1} = 1 + \frac{V_0^2}{4 E (E - V_0)}~\mathrm{sin}^2\left(\frac{2 a}{h} \sqrt{2 m (E - V_0)}\right)
\end{equation}

\section{Energies}
    \subsection{$E = V_0$}
    \subsection{$E > V_0$}
    \subsection{$E < V_0$}
    
    
    
    
\section{Physics}

Many of the symbols you will need can be found on the math page (\url{http://web.reed.edu/cis/help/latex/math.html}) and the Comprehensive \LaTeX\ Symbol Guide (enclosed in this template download).  You may wish to create custom commands for commonly used symbols, phrases or equations, as described in Chapter \ref{commands}.
