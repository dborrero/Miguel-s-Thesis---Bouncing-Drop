%The \introduction command is provided as a convenience.
%if you want special chapter formatting, you'll probably want to avoid using it altogether
		
\chapter*{Introduction}
    \addcontentsline{toc}{chapter}{Introduction}
		\chaptermark{Introduction}
		\markboth{Introduction}{Introduction}
% The three lines above are to make sure that the headers are right, that the intro gets included in the table of contents, and that it doesn't get numbered 1 so that chapter one is 1.

	(Important: Particle-wave association on a fluid interface (Protiere 2006)).
	    
	    In 2005, Yves Couder showed that bouncing oil drops on vertically vibrating fluid bath exhibited properties analogous to the paradoxical properties seen only at the quantum scale (CITE: Dynamical phenomena:  Walking and orbiting droplets?). Couder, John Bush, and others have shown that this system can reproduce double-slit single-particle interference, orbiting, tunneling, quantized orbits, spin, and more. The trajectory of the droplet can be modeled mathematically, and the dynamics of the walker have similarities to de Broglie's theory of quantum mechanics (CITE: Bush 2015).
	    
	    The literature review will begin with a description of Faraday waves and the basic dynamics of a bouncing droplet and a walking droplet. Then we will describe in detail a few of the important quantum-like properites of this system. 
	    
	    	    \subsection{Faraday Waves}
	    
	    
	?

	    
	    \subsection{Bouncing Droplets}
	    Though it had been around for at least a century, the phenomena of droplets bouncing on a vibrating fluid bath was first explained by Jearl Walker in 1978 CITE: WALKER. The first experiments looked at water droplets (bouncing on a vibrating water bath) that persisted for several seconds. Adding detergent to the water and modifying the frequency of vibration increased droplet's lifetime to minutes. Conversely any particulate impurities descrease the droplet's lifetime. Walker concluded that the droplets failed to coalesce because a layer of air trapped between the droplet and the bath would keep the two separate.\footnote{Reedie ??? wrote his thesis titled: ``???" on this very topic! } In other words, the droplet bounces on a cushion of air.
	        
	    %(NONCOALESCENCE AND NONWETTING BEHAVIOR OF LIQUIDS
%Annual Review of Fluid Mechanics
%Vol. 34: 267-289 (Volume publication date January 2002)
%DOI: 10.1146/annurev.fluid.34.082701.154240
%G. Paul Neitzel1 and Pasquale Dell'Aversana2
%)

	    
	 
	    
	    	    \subsection{Walking Droplets}
	    
	       It was Couder who then showed that an oil droplet could live for much longer. Long lifetimes meant that the focus could shift from how the doplet bounced (short time scale) to its interactions with other droplets and its motion (longer time scale).
	       
	       Every time the droplet impacts the bath, it creates a radial traveling wave. If the bouncing droplet impacts the wavefield in such a way that it recives a lateral force from the slope of the wave, then it will be pushed to the side slightly. The next time the droplet makes contanct with the bath, it will again make contact with a slope, and be pushed to the side. This propels the bouncing droplet, causing it to ``walk" across the surface of the bath. These ``walkers" turned out to have particularly interesting behaviours. Indeed, in 2006 Yves Couder and Emmanuel Fort showed that these droplets mimicked the behavior of electrons in the hallmark experiment of quantum weirdness: the double slit experiment. This was the first time that microscopic scale behavior had ever been seen at a macroscopic level, and it sparked interest in the experiment.
	    
	    