%The \introduction command is provided as a convenience.
%if you want special chapter formatting, you'll probably want to avoid using it altogether
		
\chapter*{Introduction}
    \addcontentsline{toc}{chapter}{Introduction}
		\chaptermark{Introduction}
		\markboth{Introduction}{Introduction}
% The three lines above are to make sure that the headers are right, that the intro gets included in the table of contents, and that it doesn't get numbered 1 so that chapter one is 1.





	(Important: Particle-wave association on a fluid interface (Protiere 2006)).
	    
	    In 2005, Yves Couder showed that bouncing oil drops on vertically vibrating fluid bath exhibited properties analogous to the paradoxical properties seen only at the quantum scale (CITE: Dynamical phenomena:  Walking and orbiting droplets?). Couder, John Bush, and others have shown that this system can reproduce double-slit single-particle interference, orbiting, tunneling, quantized orbits, spin, and more.

	    	    \subsection{Faraday Waves}
	    
	    
	?

	    
	    \subsection{Bouncing Droplets}
	    Though it had been seen for at least a century, the phenomena of droplets bouncing on a fluid bath was first explained by Jearl Walker in 1978\rf{Walker}. The investigations began with a simple droplet of water falling onto a bath of water, and remaining just a second too long before coalescence.\footnote{I don't drink coffee so I haven't seen this, but everyone seems to cite that this occurrs with coffeemakers, as the coffee drips into the pot.} It was then discovered that adding detergent to the water and then vibrating the bath would extend the lifetime of these droplets from fractions of a second to several minutes. Because these droplets are bouncing at frequencies of around 50 Hz (50 bounces per second) and the droplets are very small to begin with (with a diameter of a millimeter or less), it can be difficult to observe even the main mechanisms that drive the behaviour. A key insight by Walker was to flash a strobelight at a frequency slightly slower than the rate of vibration of the fluid bath, that way he could observe the droplet bouncing as if in slow motion.
	    
	    Walker found that a trapped film of air kept the droplet and the bath from touching, as shown in \refFig{bounce}. That is, the droplet is bouncing on a layer of air that's struggling to get out of the way but because the bounce happens so quickly, the fluid droplet and the fluid bath never touch. Walker concluded that the leakage rate of this trapped pocket of air depends on three factors: the nature of surface tension of the fluid bath, the viscosity of the droplet and the fluid bath, and the viscocity of the air. The bath must be of uniform surface tension and free from particulate matter floating atop the bath, since both will lead to coalescence. Higher viscosity fluids translate to longer droplet lifetimes, since more viscous fluids keep air from escaping the pocket. Finally, adjusting the frequency and the amplitudes of the vibrations also affects droplet lifetime.\footnote{Reedie Andrew Case ('92) wrote his thesis ``Oil on Troubled Water: The Extension of Floating Drop Lifetimes Due to Interface Vibration" where he looked at droplet lifetime by the frequency of vibration.}   
	     
	    More recent research suggests that a droplet fluid like silicone oil could bounce indefinately of a vibrating bath\rf{Couder2005a}. The long lifetime occurs not only because silicone oil has a high viscosity, but also because it has a \textit{low} surface tension. A low surface tension is beneficial because it makes the oil bath relatively immune to surfactants (surface acting agents) or contaminations which would otherwise make the surface tension nonuniform, and thus create a coalescence event. 
	    
	    
	    
	    	    
	   	        
	    %(NONCOALESCENCE AND NONWETTING BEHAVIOR OF LIQUIDS
%Annual Review of Fluid Mechanics
%Vol. 34: 267-289 (Volume publication date January 2002)
%DOI: 10.1146/annurev.fluid.34.082701.154240
%G. Paul Neitzel1 and Pasquale Dell'Aversana2
%)

	    
	 
	    
	    	    \subsection{Walking Droplets}
	    
	       It was Couder who then showed that an oil droplet could live for much longer. Long lifetimes meant that the focus could shift from how the doplet bounced (short time scale) to its interactions with other droplets and its motion (longer time scale).
	       
	       Every time the droplet impacts the bath, it creates a radial traveling wave. If the bouncing droplet impacts the wavefield in such a way that it recives a lateral force from the slope of the wave, then it will be pushed to the side slightly. The next time the droplet makes contanct with the bath, it will again make contact with a slope, and be pushed to the side. This propels the bouncing droplet, causing it to ``walk" across the surface of the bath. These ``walkers" turned out to have particularly interesting behaviours. Indeed, in 2006 Yves Couder and Emmanuel Fort showed that these droplets mimicked the behavior of electrons in the hallmark experiment of quantum weirdness: the double slit experiment. This was the first time that microscopic scale behavior had ever been seen at a macroscopic level, and it sparked interest in the experiment.
	    
	    