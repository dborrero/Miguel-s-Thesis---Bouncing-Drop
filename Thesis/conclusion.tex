\chapter*{Conclusion}
         \addcontentsline{toc}{chapter}{Conclusion}
	\chaptermark{Conclusion}
	\markboth{Conclusion}{Conclusion}
	\setcounter{chapter}{4}
	\setcounter{section}{0}
	
%Here's a conclusion, demonstrating the use of all that manual incrementing and table of contents adding that has to happen if you use the starred form of the chapter command. The deal is, the chapter command in \LaTeX\ does a lot of things: it increments the chapter counter, it resets the section counter to zero, it puts the name of the chapter into the table of contents and the running headers, and probably some other stuff. 

%So, if you remove all that stuff because you don't like it to say ``Chapter 4: Conclusion'', then you have to manually add all the things \LaTeX\ would normally do for you. Maybe someday we'll write a new chapter macro that doesn't add ``Chapter X'' to the beginning of every chapter title.

The question we sought to answer was: how does tunneling probability change with the value $h$ of oil above the barrier? Our results showed that tunneling is highly sensitive in this system, and even changes in height on the order of fractions of a millimeter are enough to radically influence the proportion of tunneling droplets. Using a barrier of width $e~=~3.0~\mathrm{mm}$, it was found that a value of $h~=~1.0~\mathrm{mm}$ produced tunneling in every interaction, while for a value of $h~=~1.5~\mathrm{mm}$ there was no tunneling. In between this range was a sweet spot of $h~=~1.25~\mathrm{mm}$ where tunneling appeared probabilistic, but still somewhat dependent upon droplet diameter and droplet velocity. It would appear that for a given barrier, a droplet with a higher momentum is more likely to tunnel than a droplet with lower momentum.


The main limitations in this investigation had to do with consistency of parameters between trials, and dearth of data points. A better shaker would have significantly improved these things. The damping of the shaker we used made keeping a the same memory in every trial difficult, and limited the number trials and interactions that were filmed. The one modeled in \rf{shaker}, shakes the whole tray at the same time and with same amplitude for hours. These shakers of course, cost more than the budget allowed for, but a even conducting the tests suggested in the paper would have allowed for diagnosis of these issues. Another difficulty was in measuring the height of the oil within a trial. When removing each barrier, a certain amount of oil was lost. While this value was estimated, it still was a source of uncertainty and it introduced contamination from the pliers to get into the oil. Additionally, surely there exists a better way to measure oil depth than computing the volume of space inside the tray, but a cheap alternative was not discovered. This would have also increased uncertainty in the measurements of $h$. Using barriers of height $2.90~\mathrm{mm}$ or $3.10~\mathrm{mm}$ would have allowed for greater definition in the range of tunneling heights in \refFig{tbh}.


A few words of advice for those seeking to recreate the experiment: take your time in setting up the device, ensuring that the tray is level and that it vibrates vertically. Invest in good silicone oil, and do your best to limit any contamination of the oil. Finally, there is an accelerometer out there that does what you need it to, your task is simply to find it. 